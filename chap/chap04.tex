\ifx\mainclass\undefined
\documentclass[cn,11pt,chinese,black,simple]{../elegantbook}

% 本文档命令
\usepackage{array}
\newcommand{\ccr}[1]{\makecell{{\color{#1}\rule{1cm}{1cm}}}}
\newcommand{\bfrac}[2]{\displaystyle\frac{#1}{#2}}

\makeatletter
\newcommand{\rmnum}[1]{\romannumeral #1}
\newcommand{\Rmnum}[1]{\expandafter\@slowromancap\romannumeral #1@}
\makeatother
% 示例

% 微分号
\newcommand{\dd}[1]{\mathrm{d}#1}
\newcommand{\pp}[1]{\partial{}#1}
\newcommand{\where}[1]{\Big|_{#1}}

% FT
\newcommand{\ft}[1]{\mathscr{F}[#1]}
\newcommand{\fta}{\xrightarrow{\mathscr{F}}}

% 简易图片
\newcommand{\qfig}[2]{\begin{figure}[!htb]
    \centering
    \includegraphics[width=0.6\textwidth]{#1}
    \caption{#2}
  \end{figure}}

% 表格
\renewcommand\arraystretch{1.5}

\begin{document}
\fi 

% Start Here
\chapter{载流子输运过程}


产生和复合过程是在能带图上的垂直向关系,输运则是在能带图的水平向的运动。输运的增强有两种主要的形式:漂移与扩散。

\section{漂移}

漂移电流的定义 

\[J_n = q n v_{drift} = q n \mu_n \mathscr{E}\]

其中 \(\mu_n\) 是迁移率。

在外场的作用下,方程中的质量使用有效质量代替来等效内场

\[\frac{d\left(m_{n}^{*} v\right)}{d t}=-q E -\frac{m_{n}^{*} v}{\tau_{n}}\]

\[\begin{array}{l}
    v(t)=-\frac{q \tau_{n}}{m_{n}^{*}} E \left[1-e^{-\frac{t}{\tau_{n}}}\right] \\
    =-\frac{q \tau_{n}}{m_{n}^{*}} E \quad(t \rightarrow \infty, 1-2 ps ) \\
    \equiv \mu_{n} E
\end{array}\]

\section{迁移率}

迁移率与散射时间满足 

\[\mu_n = \dfrac{q \tau_n}{m_n^*}\]

在仅考虑电离的杂质时

\[\tau_n \sim \dfrac{T^{3/2}}{N_D}\]

在高温情况下,声子散射成为主导时

\[\tau_n \sim T^{-3/2}\]

对于不同种类的迁移率,总迁移率相当于并联。

\section{高场效应}

实际上就是速度饱和,之前的 \(v = \dfrac{q \tau_N}{m_N^*} \mathscr{E}\),在 \(\mathscr{E}\) 超过某个阈值之后,速度不再继续变化。

但是 GaAs 存在速度过冲现象,在提供的能量足够大的情况下,可以越到另一个能谷中。

\section{扩散电流}

\(D_P, D_N\)分别是空穴、电子的扩散系数,单位是 \(\text{cm}^2/\text{sec}\)

\[\begin{array}{l}
    \left. J _{ P }\right|_{\text {diff }}=-q D_{ P } \nabla p \\
    J _{ N \mid \text { diff }}=q D_{ N } \nabla n
\end{array}\]

\section{霍尔效应}

对于某个材料的电阻率\(\rho\)有

\[\mathscr{E} = \rho J = \rho q (\mu_n n + \mu_p p) \mathscr{E}\]

那么 

\[\rho = \dfrac{1}{q (\mu_n p + \mu_p p)}\]

那么对于 n 型半导体 \(\rho = \dfrac{1}{q \mu_n N_D}\) ,对于 p 型半导体 \(\rho = \dfrac{1}{q \mu_p N_A}\) 。

Drude 模型

\[-q E -q v \times B -\frac{m^{*} v}{\tau}=0\]

解得

\[\begin{array}{l}
    \begin{aligned}
        m^{*} v &=-q \tau \mathscr{E} -q \tau v \times B \\
        &\approx-q \tau \mathscr{E} -q \tau\left(-\frac{q \tau \mathscr{E} }{m^{*}}\right) \times B \\
        &=-q \tau \mathscr{E} +\frac{q^{2} \tau^{2}}{m^{*}} \mathscr{E} \times B \\
        v&=-\frac{q \tau \mathscr{E} }{m^{*}}+\frac{q^{2} \tau^{2}}{m^{* 2}} \mathscr{E} \times B
    \end{aligned}
\end{array}\]

在弱磁场下

\[-q \mathscr{E} - \dfrac{m^* v}{\tau} \approx 0\]

解得 

\[v' = \dfrac{-q \tau \mathscr{E}}{m^*}\]

对于一般的电流 

\[\begin{aligned}
    J _{n} &=-q n v \\
    &=\frac{q^{2} n \tau}{m^{*}} \mathscr{E} -\frac{q^{2} n \tau}{m^{*}} \frac{q \tau}{m^{*}} \mathscr{E} \times B \\
    &=\sigma_{0} \mathscr{E} -\sigma_{0} \mu \mathscr{E} \times B
\end{aligned}\]


解得霍尔电阻

\[\left[\begin{array}{c}
    J_{x} \\
    J_{y}
    \end{array}\right]=\left[\begin{array}{cc}
    \sigma_{0} & -\sigma_{0} \mu B_{z} \\
    \sigma_{0} \mu B_{z} & \sigma_{0}
    \end{array}\right]\left[\begin{array}{c}
    E_{x} \\
    E_{y}
\end{array}\right]\]

对于平衡态,\(J_y = 0\) \(B_z \approx 0\)

\[\left[\begin{array}{c}
    J_{x} \\
    0
    \end{array}\right]=\left[\begin{array}{cc}
    \sigma_{0} & 0 \\
    \sigma_{0} \mu B_{z} & \sigma_{0}
    \end{array}\right]\left[\begin{array}{c}
    E_{x} \\
    E_{y}
\end{array}\right]\]

解得

\[R_H = \dfrac{E_y/B_z}{J_x} = - \dfrac{1}{q n}\]

\(R_H\)通常在0.5 到 2 之间

\section{连续性方程}

共有五组

\[\begin{array}{l}
    \frac{\partial n}{\partial t}=\frac{1}{q} \nabla \bullet J _{N}-r_{N}+g_{N} \\
    J _{N}=q n \mu_{N} E+q D_{N} \nabla n \\
    \frac{\partial p}{\partial t}=-\frac{1}{q} \nabla \bullet J _{P}-r_{P}+g_{P} \\
    J _{P}=q p \mu_{P} E-\left(q D_{P} \nabla p\right) \\
    \nabla \bullet D=q\left(p-n+N_{D}^{+}-N_{A}^{-}\right)
\end{array}\]

其中扩散项中的扩散系数满足爱因斯坦关系,体现了散射在载流子的扩散用户漂移中均起了主导的作用

\[\dfrac{D}{\mu} = \dfrac{k_B T}{q}\]

\subsection{平衡态}

满足

\[J _{N}=q n \mu_{N} E+q D_{N} \dfrac{\dd{n}}{\dd{x}} \]

解得 

\[\frac{1}{n} \frac{\dd{n}}{\dd{x}} = - \frac{\mu_n \mathscr{E}}{D_N}\]

那么在势场的不同位置有

\[n_{2}=n_{1} e^{-\int_{0}^{L} \frac{\mu_{n}}{D_{n}} \mathscr{E}}{=n_{1} e^{\frac{\mu_{n} V}{D_{n}}}}\]

为了满足爱因斯坦关系,必然有两点的费米能级相等

\[\frac{n_{2}}{n_{1}}=\frac{N_{C} e^{-\left(E_{C 2}-E_{F}\right) / k T}}{N_{C} e^{-\left(E_{C 1}-E_{F}\right) / k T}}=e^{-\left(E_{C 2}-E_{C 1}\right) / k T}=e^{q V / k T}\]

\subsection{载流子电流}

\begin{equation*}
    \begin{array}{l}
        J _{N}=q n \mu_{N} E+q D_{N} \nabla n \\
        J _{P}=q p \mu_{P} E-\left(q D_{P} \nabla p\right) 
    \end{array}
\end{equation*}

第一项是漂移电流,第二项是扩散电流。

\subsection{连续性方程}


\[\begin{array}{l}
    \frac{\partial n}{\partial t}=\frac{1}{q} \nabla \bullet J _{N}-r_{N}+g_{N} \\
    \frac{\partial p}{\partial t}=-\frac{1}{q} \nabla \bullet J _{P}-r_{P}+g_{P} \\
\end{array}\]

第一项是漂移、扩散共同引起的载流子变化,第二项是产生-复合引起的,最后是其他效应。


% End Here

\ifx\mainclass\undefined
\end{document}
\fi 