\ifx\mainclass\undefined
\documentclass[cn,11pt,chinese,black,simple]{../elegantbook}
\usepackage{array}

% 本文档命令
\usepackage{array}
\newcommand{\ccr}[1]{\makecell{{\color{#1}\rule{1cm}{1cm}}}}
\newcommand{\bfrac}[2]{\displaystyle\frac{#1}{#2}}

\makeatletter
\newcommand{\rmnum}[1]{\romannumeral #1}
\newcommand{\Rmnum}[1]{\expandafter\@slowromancap\romannumeral #1@}
\makeatother
% 示例

% 微分号
\newcommand{\dd}[1]{\mathrm{d}#1}
\newcommand{\pp}[1]{\partial{}#1}
\newcommand{\where}[1]{\Big|_{#1}}

% FT
\newcommand{\ft}[1]{\mathscr{F}[#1]}
\newcommand{\fta}{\xrightarrow{\mathscr{F}}}

% 简易图片
\newcommand{\qfig}[2]{\begin{figure}[!htb]
    \centering
    \includegraphics[width=0.6\textwidth]{#1}
    \caption{#2}
  \end{figure}}

% 表格
\renewcommand\arraystretch{1.5}

\begin{document}
\fi 

% Start Here
\chapter{态密度}

\section{态密度关系的推导}

在间距为\(\Delta k\)的两个能态之间存在的态的个数为\(2 \Delta k / \delta k\) 其中 \(\delta k = 2 \pi / a / N\) ,即为第一布里渊区的相邻态间距。
那么单位能量间隔的态为 \(\delta k = 2 \pi / a / N/ \Delta E\)

而能量满足\[E-E_0 = \bfrac{\hbar^2 k^2}{2 m^*}\]
 
那么有\[\bfrac{\dd{k}}{\dd{E}} \sqrt{\bfrac{m^*}{2 \hbar^2 (E-E_0)}}\]

对于含有 \(N\) 个原子的一维链来说 \(L = N a\) ,联立,那么单位能量单位长度的态密度为

\[DOS = \bfrac{1}{\pi}\bfrac{\dd{k}}{\dd{E}} \sqrt{\bfrac{m^*}{2 \hbar^2 (E-E_0)}} \]

\qfig{f1.png}{一维态密度}

类似的对于三维体系,在 \(E\) 处的能量间隔中态的数目为 
\[\bfrac{\Delta V}{\delta V} = \bfrac{4 \pi /3 [(k + \dd{k})^3 - k^3 ]}{(2 \pi / L)(2 \pi / W)(2 \pi / H)} = \bfrac{V}{2 \pi^2 k^2 \dd{k}}\]

同一维情况的 \(E-k\)关系带入

\[DOS = \bfrac{m^*}{2 \pi^2 \hbar^3} \sqrt{2 m* (E-E_0)}\]

\section{常见材料的导带价带}

关键在于求得等效质量\(m^*\)

\[g_i(E) = \bfrac{m^*}{2 \pi^2 \hbar^3} \sqrt{2 m* (E-E_0)}, i = c\  \text{or}\  v\]

其中有效质量为

\[
\bfrac{1}{m^*} = \bfrac{1}{\hbar^2} \bfrac{\dd{^2E}}{\dd{k^2}}  
\]

由于不同材料的能带形状不一样,如椭球型,需要求得其等效质量\(m^*_{eff}\)。

GaAs 的导带为球形,各个方向的有效质量一致。

Si 与 Ge 的导带形状为椭球形,满足:

\[
E - E_C = \bfrac{\hbar^2 k_1^2}{2 m_l ^*} +\bfrac{\hbar^2 k_2^2}{2 m_t ^*} + \bfrac{\hbar^2 k_1^2}{2 m_t ^*} 
\]

其中\(m_l^*\)是椭球的长轴等效质量,\(m_t^*\)是短轴的。

可以归一化为关于\(k_i\)的椭球面方程,

\[1 = \bfrac{k_1^2}{\alpha^2} + \bfrac{k_2^2}{\beta^2} + \bfrac{k_3^2}{\beta^2}\]

将第一布里渊区的有效椭球面个数记为 \(N_{el}\) ,进而定义有效半径

\[N_{el} \bfrac{4}{3} \pi \alpha\beta^2 = \bfrac{4}{3} \pi k_{eff}^3\]

化简得到有效质量

\[m_{eff}^* = N_{el}^{2/3}  (m_l^* m_t^{*2})^{1/3}\]

对于 Ge \(N_{el} = 4\) ,Si \(N_{el} = 6\) 。

对于价带,同样由不同的有效质量如重空穴与轻空穴分别对应 \(m^*_{hh} m^*_{lh}\)。





\section{费米-狄拉克分布}

费米能级为\(E_F\)

\[f_0(E) = \bfrac{1}{1+e^{\beta (E- E_F)}}\]

其中 \(\beta = 1 / (k_B T)\)

\section{载流子分布}

电子的单位体积个数(浓度)可以考虑成导带电子态密度以及电子占据情况下分布函数的加权

\[n = \int_{E_c}^{E_{top}} g_c(E) f(E) dE\]

空穴的单位体积个数(浓度)可以考虑成价带空穴态密度以及电子\textbf{不}占据情况下分布函数的加权

\[p = \int_{E_bottom}^{E_{v}} g_v(E) (1 - f(E)) dE\]



\section{玻尔兹曼分布}

在导带或者价带能量与费米能级相差\(3 k_B T\) 及以上时, 费米-狄拉克分布可以弱化为玻尔兹曼分布,并且可以称为简并半导体。

\[n = N_C  e^{-\beta(E_c - E_F)}\]


\[p = N_V  e^{\beta (E_v - E_F)}\]



其中,\(N_C\) 、 \(N_V\) 分别是导带价带的有效态密度。

\begin{equation*}
    \begin{aligned}
        N_C = 2 \left(\bfrac{2 \pi m_n^* k_B T}{h^2}\right) ^{3/2} \\
        N_V = 2 \left(\bfrac{2 \pi m_p^* k_B T}{h^2}\right) ^{3/2} 
    \end{aligned}
\end{equation*}

\section{本征费米能级}

由上一节知

\[n p = N_C N_V e^{-\beta(E_c - E_v)} =N_C N_V e^{-\beta(E_g)} \]

当\(n = p = n_i\) 时,定义 \(E_F = E_i\) 
可以解出本征载流子浓度与费米能级

\[n_i = \sqrt{N_C N_V} e^{-E_0/2k_B T}\]

\[E_i = \dfrac{E_G}{2} + \dfrac{1}{2 \beta} \ln \dfrac{N_V}{N_C}\]

并且存在关系

\[\begin{array}{l}
    n_{ i }=N_{ C } e^{\left(E_{ i }-E_{ c }\right) / k T} \\
    n_{ i }=N_{ V } e^{\left(E_{ v }-E_{ i }\right) / k T}
\end{array}\]

\[\begin{array}{l} 
    n = n _{ i } e^{\left(E_{ F }-E_{ i }\right) / k T} \\
    p = n _{ i } e^{\left(E_{ i }- E _{ F }\right) / k T}
\end{array}\]

\section{能带图与能量}

以一个电子的能量分布为例,空穴类似。

\qfig{f2.png}{电子的能量分布}

\qfig{f3.png}{电子与空穴的对比}

可以得到势能满足 

\[P.E = E_c - E_{ref} = -q V(x)\] 

那么 \[V = - \dfrac{1}{q}(E_c - E_{ref})\]

电场满足 \[\mathscr{E} = -\nabla V = \dfrac{1}{q}\dfrac{\dd{E_c}}{\dd{x}}   = \dfrac{1}{q}\dfrac{\dd{E_v}}{\dd{x}}\]

% End Here

\ifx\mainclass\undefined
\end{document}
\fi 