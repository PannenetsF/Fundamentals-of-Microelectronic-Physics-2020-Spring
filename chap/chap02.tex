\ifx\mainclass\undefined
\documentclass[cn,11pt,chinese,black,simple]{../elegantbook}
\usepackage{array}


% 本文档命令
\usepackage{array}
\newcommand{\ccr}[1]{\makecell{{\color{#1}\rule{1cm}{1cm}}}}
\newcommand{\bfrac}[2]{\displaystyle\frac{#1}{#2}}

\makeatletter
\newcommand{\rmnum}[1]{\romannumeral #1}
\newcommand{\Rmnum}[1]{\expandafter\@slowromancap\romannumeral #1@}
\makeatother
% 示例

% 微分号
\newcommand{\dd}[1]{\mathrm{d}#1}
\newcommand{\pp}[1]{\partial{}#1}
\newcommand{\where}[1]{\Big|_{#1}}

% FT
\newcommand{\ft}[1]{\mathscr{F}[#1]}
\newcommand{\fta}{\xrightarrow{\mathscr{F}}}

% 简易图片
\newcommand{\qfig}[2]{\begin{figure}[!htb]
    \centering
    \includegraphics[width=0.6\textwidth]{#1}
    \caption{#2}
  \end{figure}}

% 表格
\renewcommand\arraystretch{1.5}

\begin{document}
\fi 

% Start Here

\chapter{掺杂}

\section{基础知识}

施主原子 (Donor) :常见 \Rmnum{5} 族元素,掺杂后产生 n 型半导体

受主原子 (Acceptor) : 常见 \Rmnum{3} 族元素,掺杂后产生 p 型半导体

掺杂会使得禁带附近出现能级,使得禁带宽度变窄。

\qfig{f4.png}{掺杂引起的能级变化}

\section{施主能级与受主能级的统计分布}

在施主能级上为空的概率,也就是施主的电子变成载流子的概率, \[\dfrac{N_D^+}{N_D} = 1 - f(E_D) =\dfrac{1}{1 + g_D e^{(E_F - E_D)/k_B T}}\]


在受主能级上占据的概率,也就是受主的空穴变成载流子的概率 \[\dfrac{N_A^-}{N_A} = f(E_A) = \dfrac{1}{1 + g_A e^{(E_A - E_F)/k_B T}}\]


其中\(g_D, g_A\)是能级的简并度,通常分别取\(2,4\)。可以将其转换为 \(e^{\epsilon / k_B T}\) 的形式,使得整体保持与之前费米狄拉克分布形式的一致,修正后的能级称为等效能级。

\begin{equation*}
    \begin{aligned}
        E_D' = E_D - \epsilon \\
        E_A' = E_A + \epsilon
    \end{aligned}
\end{equation*}


其中\(N_D\)是总的掺杂施主浓度,\(N_D^+\) 是电离的施主浓度。\(N_A\)是总的掺杂受主浓度,\(N_A^-\) 是电离的受主浓度。

% 各个物理量的意义

电荷密度为 
\[\rho = q (p - n + N_D^+ - N_A^-)\]

电中性条件 \[\int_{V} \rho \dd{V} = 0 \]

\begin{equation}\label{eq:ch2:1}
    \begin{aligned}
        N_Ve^{-(E_F-E_V)/k_B T} &- N_A e^{-(E_C - E_F)/k_B T} \\
        & + \dfrac{N_D}{1 + 2 e^{(E_F - E_D)/k_B T}} - \dfrac{N_A}{1 + 4 e^{E_A - E_F}/k_B T } = 0
    \end{aligned}
\end{equation}

从掺杂的角度认识本征载流子浓度即为:电离的掺杂离子浓度均为 \(0\) ,此时 \(n = p = n_i\) 。



\section{载流子浓度与温度的关系}

本节以及之后几节都以施主掺杂为例,受主掺杂类似。

在温度接近 \(0 K\) 时,几乎不会出现带 - 带之间的跃迁,大部分载流子由电离提供。
在室温情况下,掺杂离子几乎完全电离,但是跃迁仍然较少,载流子浓度呈现为稳定的状态。再升高温度,材料本身的载流子被激发出来,最终超过掺杂原子电离出的载流子成为主要贡献者。

\qfig{f6.png}{温度与跃迁}

\qfig{f5.png}{载流子浓度与温度关系}

\section{平衡态的载流子浓度}

~\ref{eq:ch2:1} 中,\(N_A^- = 0\) ,即

\[p - n + N^+_D = 0\]

对于 \(N_D^+ \) 

\[n = N_C e^{-\beta (E_C - E_F)}\]

那么

\[e^{\beta E_F} = \dfrac{n}{N_C}e^{\beta E_C}\]

\[N_D^+ =\dfrac{N_D}{1 + 2 e^{(E_F - E_D)/k_B T}} = \dfrac{N_D}{1 + 2 \left[\dfrac{n}{N_C}e^{\beta (E_C - E_D} \right]} \equiv \dfrac{N_D}{1 + \dfrac{n}{N_{\xi}}}\] 

其中 \(N_\xi \) 是对特定温度可以计算的常数

\[N_\xi \equiv (N_C / g_D) e^{-(E_c - E_D)/k_B T}\]

同时 \(n p = n)i^2\) 式~\ref{eq:ch2:1} 化简为

\[\dfrac{n_i^2}{n} - n + \dfrac{N_D}{\dfrac{n}{N_\xi} + 1} = 0\]

按照上一节讨论的温度分类进行讨论

\begin{enumerate}
    \item 低温:\(N_D \gg n_i\) \[- n + \dfrac{N_D}{\dfrac{n}{N_\xi} + 1} = 0\]
    \[n = \dfrac{N_\xi}{2} \left[\left(1 + \dfrac{4N_D}{N_\xi}\right)^{1/2} -1\right]\]
    \item 室温:上式仍成立 但是此时 \(N_\xi \gg N_D\) ,泰勒展开近似得到
    \[n = N_D\]
    \item 高温:此时完全电离 \(N_D^+ = N_D\) 而且 \(n_i\) 已经无法忽视,\(N_\xi\) 相当大。
    \[\dfrac{n_i^2}{n} - n + N_D = 0\]
    \[n = \dfrac{N_D}{2} + \left[\dfrac{N_D^2}{4} + n_i^2\right]^{1/2}\]
    满足
    \begin{equation*}
        n \approx \left\{ 
        \begin{aligned}
            N_D, & N_D \gg n_i \\
            n_i, & n_i \gg N_D
        \end{aligned}\right.
    \end{equation*}
\end{enumerate}

\section{费米能级的确定}

在本征状态下,即无掺杂时,\(E_F = E_i\)满足

\[
\begin{aligned}
    n &= p \\
    N_C e^{-\beta (E_c - E_F)} &= N_V  e^{+ \beta (E_v - E_F)}
\end{aligned}    
\]

解得 

\[
E_F = E_i = \dfrac{E_G}{2} + \dfrac{1}{2\beta} \ln \dfrac{N_V}{N_C}
\]

这里 \(E_G = \dfrac{E_c + E_v}{2}\)

在室温掺杂下 \(n = N_D\) ,那么有

\[n=N_{C} e^{-\beta\left(E_{c}-E_{F}\right)} \Rightarrow E_{F}=E_{C}+\dfrac{1}{\beta} \ln \left(\dfrac{n}{N_{C}}\right)\]

在非本征状态下,由上一章的公式 

\[\begin{array}{l}
    E_{ F }-E_{ i }=k T \ln \left(N_{ D } / n_{ i }\right) \\
    E_{ i }-E_{ F }=k T \ln \left(N_{ A } / n_{ i }\right)
\end{array}\]

% End Here

\ifx\mainclass\undefined
\end{document}
\fi 